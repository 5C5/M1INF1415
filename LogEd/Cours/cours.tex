%Espace pour le cours du 10/03

\Chapter{Fonctionnalités des logicielles éducatifs}
Ce qui rend un système intelligent

\begin{itemize}
	\item Modélisation du domaine : permet de représenter les connaissances(celles déjà connues par les apprenants et celles qu'ils doivent acquérir) et manipuler les objets du domaine\\
	Il faut savoir doser les connaissances pour qu'elles correspondent à l'environnement de l'apprenant (son âge, son niveau)\\
	Trois niveaux de représentations :
	\begin{itemize}
		\item Représentations mentale de l'apprenant
		\item Représentations internes destinées au système
		\item représentations externes destinées à la communication entre le système et l'apprenant, c'est c'elle que l'on voit, qu'on manipule dans le logiciel (IHM)
	\end{itemize}
	L'environnement logiciel n'est pas identique au monde réel => différentes représentations possibles, chaque personne ayant sa représentatin propre\\
	Sytème à base de connaissance -> le logiciel doit savoir résoudre les problèmes qu'il pose à l'apprennant. Possibilité de proposer des réponses alternatives, des indications en cas de réponses fausses, en s'adaptant (résolveur pédagogique). On s'intéresse aux connaissances à acquérir ≠ connaissances de l'expert. On manipule les connaissances en s'adaptant au niveau de l'utilisateur.
	\item Résolveur : Résoudre les problèmes, évaluer les réponses au fur et à mesure, indentifier toutes les solutions correctes

	\item Générateur d'exercice
	\item Modèle/Profil de l'apprenant : utile pour plein de chose, permet de personaliser l'apprentissage (point de vue du système/de l'enseignant sur l'apprenant, permet au système de s'adapter aux connaissances de l'apprenant)
	\item Aide : intervention demandée par l'apprenant, proposée par le système (spontanément, donc détection de difficulté), à adapter à l'apprenant et au contexte. L'aide peut être un rappel des consignes, des indications sur l'interface, ou des explications sur un point précis...
	\item Explications : sur les connaissances du domaine, sur la résolution du problème, sur les erreurs de l'apprenant, sur la conduite de session (activité, objectif pédagogique). Elles s'appuient sur les connaissances dur résolveur, les solutions identifiées, la réponse de l'apprenant, des méta-connaissances pour expliquer les stratégies de résolution.
	\item Module pédagogique : Gestio du parcours de l'apprenant : que proposer, à qui, à quel moment, comment... Choix des problèmes et des activités, choix des aides et explications...

\end{itemize}

\vskip 4cm

\section{Personnalisation de l'apprentissage}

Deux approches pour la personnalisation :
\begin{itemize}
	\item La personnalisation par les souhaits de l'apprenant. On identifie quels sont les besoins de l'apprenant, c'est l'apprenant qui décide et choisit
	\item La personnalisation par les compétences de l'apprenant. On identifie les compétences de l'apprenant( celles qu'il a, non pas celles qu'il prétend avoir). On utilise des tests pour identifier ses connaissances à un grain moyen, et déterminer la connaissance suivante à enseigner à l'apprenant.
\end{itemize}


\vskip 1cm
On étudie ce que sait l'apprenant, on fait un profil de l'apprenant, car
\begin{itemize}
	\item On ne peut enseigner que à ceux qui savent déjà : on apprend à partir de ce qu'on sait (zone proximale de l'apprenant). On doit adapter son enseignement à l'apprenant
	\item Traitement des erreurs : il faut détecter les erreurs, mais aussi comprendre ses causes et empêcher qu'elles se reproduisent.
	\item Bilan de compétences :
	\item Assistance à l'enseignant
\end{itemize}

Besoin d'un représentation des connaissances de l'apprenant.
La solution pédagogique (le logiciel éducatif, par exemple) part de connaissances initiales et amènes des connaissances résultantes, qui soit se rajoutent aux connaissances initiales, soit modifient certaines des connaissances initiales.\\
Une partie importante des connaissances se trouvent dans la mise en relaion des connaissances.


\subsection{Profil : définition}
Ensemble d'informations
\begin{itemize}
	\item concernant l'apprentissage de l'apprenant(connaissances théoriques, factuelles, erronées,conceptions,  compétences, savoir-faire, habiletés, comportements)
	\item Valuées (selon des échelles), par des valeurs numériques, textuelles graduées, textuelles non graduées, etc
	\item collectées ou déduites à l'issue d'activités pédagogiques (informatisées, papier-crayon). 
\end{itemize}
Des structures de données pour représenter l'état des connaissance de l'apprenant dans le domaine considéré, du pt de vue du système

\vskip 1cm

différence entre modèle, profil et diagnostic

\vskip 1cm

6 fnct° du modèle de l'apprennant :
\begin{itemize}
	\item correction -> aider à corriger des connaissances érronées
	\item élaboration -> déterminer le point à étudier ensuite
	\item stratégie -> déterminer la stratégie pédagogique, type d'intervention du système
	\item diagnostic -> vérifier et préciser le modèle de l'élève
	\item prédiction -> prédire le comportement futur de l'apprenant
	\item évaluation -> évaluer ou mesurer l'efficacité
\end{itemize}
Plus, le modèle ouvert : l'apprenant a accès à son profil.\\

\subsection{constitution du profil de l'étudiant}
Collecte des informations, obtention d'observables, diagnostic pour l'interprétation des observables, création du profil de l'apprenant.

Observation de l'apprenant, construction de tests spécifiques pour identifier ses connaissances.\\
Utilisation de sources d'informations explicites (questions posées à l'apprenant), implicites(analyse de l'iteraction de l'apprenant avec le système), structurel(parcours de l'apprenant, pas bien, mais mieux que rien), valeurs par défauts, profils stéréotyés.\\

Évolution du modèle de l'apprenant, car non certain, connaissances évoluant (donnant lieu à des incohérences dans le modèle de l'apprenant)\\

Comparaison des connaissances aux connaissances attendues.\\

Usagers du profil de l'apprenant :
\begin{itemize}
	\item par le logiciel qui l'a créé, ou un autre log educatif, personnalisation du parcours, du ryhtme, des contenus, des interractions
	\item Les acteurs humains dans l'environnement de l'apprenant
\end{itemize}


\end{document}
