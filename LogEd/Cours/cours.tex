%Espace pour le cours du 10/03

\Chapter{Fonctionnalités des logicielles éducatifs}
Ce qui rend un système intelligent

\begin{itemize}
	\item Modélisation du domaine : permet de représenter les connaissances(celles déjà connues par les apprenants et celles qu'ils doivent acquérir) et manipuler les objets du domaine\\
	Il faut savoir doser les connaissances pour qu'elles correspondent à l'environnement de l'apprenant (son âge, son niveau)\\
	Trois niveaux de représentations :
	\begin{itemize}
		\item Représentations mentale de l'apprenant
		\item Représentations internes destinées au système
		\item représentations externes destinées à la communication entre le système et l'apprenant, c'est c'elle que l'on voit, qu'on manipule dans le logiciel (IHM)
	\end{itemize}
	L'environnement logiciel n'est pas identique au monde réel => différentes représentations possibles, chaque personne ayant sa représentatin propre\\
	Sytème à base de connaissance -> le logiciel doit savoir résoudre les problèmes qu'il pose à l'apprennant. Possibilité de proposer des réponses alternatives, des indications en cas de réponses fausses, en s'adaptant (résolveur pédagogique). On s'intéresse aux connaissances à acquérir ≠ connaissances de l'expert. On manipule les connaissances en s'adaptant au niveau de l'utilisateur.
	\item Résolveur : Résoudre les problèmes, évaluer les réponses au fur et à mesure, indentifier toutes les solutions correctes

	\item Générateur d'exercice
	\item Modèle/Profil de l'apprenant : utile pour plein de chose, permet de personaliser l'apprentissage (point de vue du système/de l'enseignant sur l'apprenant, permet au système de s'adapter aux connaissances de l'apprenant)
	\item Aide : intervention demandée par l'apprenant, proposée par le système (spontanément, donc détection de difficulté), à adapter à l'apprenant et au contexte. L'aide peut être un rappel des consignes, des indications sur l'interface, ou des explications sur un point précis...
	\item Explications : sur les connaissances du domaine, sur la résolution du problème, sur les erreurs de l'apprenant, sur la conduite de session (activité, objectif pédagogique). Elles s'appuient sur les connaissances dur résolveur, les solutions identifiées, la réponse de l'apprenant, des méta-connaissances pour expliquer les stratégies de résolution.
	\item Module pédagogique : Gestio du parcours de l'apprenant : que proposer, à qui, à quel moment, comment... Choix des problèmes et des activités, choix des aides et explications...

\end{itemize}

\vskip 4cm

