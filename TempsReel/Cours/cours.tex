\documentclass{article}

\title{Cours de Système en temps réel\\(enrichie des commentaires du prof)}

\begin{document}

\section{notions :}
	\begin{itemize}
	\item QoS (Qualité de Service)
	\item Temps réel ≠ rapide
	\item Échéance : moment ou le résultat doit avoir été calculé
	\item Temps réel dur (Hard real time) : on respecte toutes les échéances.
	\item Temps réel lache (soft real time) : on peut rater execptionnellement des échéances
	\item Déterminisme : on sait à l'avance combien de temps au maximum cela peut prendre
	\item Nécessaire : quand on a des actionneurs et capteurs
	\item exemple de temp réel
	\begin{itemize}: 
		\item gravage de CD -> nécessité de tout reprendre du début en cas de problème car peu de ram dans les lecteurs de cd, donc forcément temps réel
		\item carte son : immédiateté des sons : tout petit buffer à remplir progressivement (petit buffer à cause de la latence)
		\item GSM : compression du son, par petit paquets
		\item routeur réseau -> veut faire passer des paquets le plus vite possible d'un point à un autre
	\end{itemize}
\end{itemize}

Système Temps Réel(STR) et Système Généraliste (SG)
SG : 
\begin{itemize}
	\item Objectif : que ça aille le plus vite possible. On optimise le cas moyen
\end{itemize}
STR :
\begin{itemize}
	\item Objectif : respecter son échéance. On s'intéresse au pire cas, à combien de temps cela prendra dans le pire des cas. On optimise donc le pire cas.
	\item Algorithmes : 
	\begin{itemize}
		\item Pas de Quicksort (O(n²) dans le pire cas); 
		\item Jamais utiliser de cache : on considère que les calculs sont tous refaits à chaque fois
		\item Pas de table de hachage : Dans le pire cas, tous les éléments de la table se retrouvent au même endroit
		\item peu de switch(C) car algo linéaire -> le compilateur lit les cas les uns après les autres
	\end{itemize}
\end{itemize}

Qu'est ce qui empêche un système d'être en temps réel?
\begin{itemize}
	\item Temps CPU
	\item les entrées sorties
	\item les interruptions et exceptions
	\item les zones critiques (zones à verrous, forçant des processus à attendre si déjà accédées)
\end{itemize}

Au démarage, le processeur reçoit une interruption Reset -> l'adresse mémoire du Reset contenue dans le vecteur d'inerruption (liste d'adresse mémoire) est stocké dans le compteur ordinal(registre contenant l'@ de la prochaine instruction à éxécuter). Ce programme est le BIOS (Basic Input Output System). Il démarre un autre programme intermédiaire, généralement très court (GRUB, LILO), qui charge tout le système en mémoire et démarre son exécution.
Le système ((Linux) Noyau + Initrd/Initramfs (système fichier contenant le nécessaire pour faire démarrer le système)) initialise la mémoire. Puis, il initialise le système de fichiers, la mémoire virtuelle (définie par le Memory Management Unit) et l'ordonanceur. Lancement de init, qui démarre la machine. Historiquement, init regarde un fichier nommé /etc/inittab et /etc/init.d/...  (/etc/rc2.d/... -> lien symboliques vers /etc/init.d/..). Lancement de udev qui gère les périphériques.


Temps Réel Minimaliste
(une intro pour montrer qu'on peut pas s'en sortir comme ça)

POSIX 1003.1b norme se résumant à une bibliothèque de fonctions minimales (minimales minimales, genre du sémaphore, le pipe, des timer et des interruptions) pour écrire des applications. Cela représente quelques ko de code. Ce sont des fonctions avec un comportement déterministe.

Boucle de scrutation : le programme regarde les entrées, fait les actions, et recommence.
On veut pouvoir avoir des actions longues -> thread -> verrous car sections critiques.
Faire le temps réel dans le gestionnaire d'interruption -> comme le fait le SE.

On veut un système complet temps réel. -> faire des programmes les plus simples possibles


\section{Système d'exploitation}
\subsection{Interruption}
\begin{itemize}
	\item périphérique (DMA, données à traiter). Tous les dialogues avec des périphériques se font via des zones mémoires.
	\item Défaut de page (MMU)
	\item Real Time Clock, composant programmable envoyant des interruptions à des intervalles ou après des durées choisies. Ce composant est extrêment lent à programmer, donc on y touche au minimum
	\item Exception (±∞, Nan, /ø)
\end{itemize}

\subsection{Que se passe t'il quand on reçoit une interruption du RTC?}
Une interruption déroute le cpu.
\begin{enumerate}
	\item Le processeur sauvegarde le compteur ordinal et le registre de status
	\item le processus passe en mode root
	\item On met l'@ du code indiqué dans le vecteur d'interruption dans le compteur ordinal
	\item gestionnaire d'interruption : ordonnanceur
	\item sauvegarde des registres que l'on va modifier (ceux qui contiennent des entiers)
	\item Si on change de tâche 
	\begin{enumerate}
		\item Sauvegarde la sauvegarde du compteur ordinal et du registre de status dans la taille d'information, ainsi que les autres registres (+ les registres flotantes)
		\item On récupère les mêmes informations du nouveau processusCours de Système en temps réel(enrichie des commentaires du prof)
	\end{enumerate}
\end{enumerate}


\section{notions :}
	\begin{itemize}
	\item QoS (Qualité de Service)
	\item Temps réel ≠ rapide
	\item Échéance : moment ou le résultat doit avoir été calculé
	\item Temps réel dur (Hard real time) : on respecte toutes les échéances.
	\item Temps réel lache (soft real time) : on peut rater execptionnellement des échéances
	\item Déterminisme : on sait à l'avance combien de temps au maximum cela peut prendre
	\item Nécessaire : quand on a des actionneurs et capteurs
	\item exemple de temp réel
	\begin{itemize}: 
		\item gravage de CD -> nécessité de tout reprendre du début en cas de problème car peu de ram dans les lecteurs de cd, donc forcément temps réel
		\item carte son : immédiateté des sons : tout petit buffer à remplir progressivement (petit buffer à cause de la latence)
		\item GSM : compression du son, par petit paquets
		\item routeur réseau -> veut faire passer des paquets le plus vite possible d'un point à un autre
	\end{itemize}
\end{itemize}

Système Temps Réel(STR) et Système Généraliste (SG)
SG : 
\begin{itemize}
	\item Objectif : que ça aille le plus vite possible. On optimise le cas moyen
\end{itemize}
STR :
\begin{itemize}
	\item Objectif : respecter son échéance. On s'intéresse au pire cas, à combien de temps cela prendra dans le pire des cas. On optimise donc le pire cas.
	\item Algorithmes : 
	\begin{itemize}
		\item Pas de Quicksort (O(n²) dans le pire cas); 
		\item Jamais utiliser de cache : on considère que les calculs sont tous refaits à chaque fois
		\item Pas de table de hachage : Dans le pire cas, tous les éléments de la table se retrouvent au même endroit (En donnant des valeurs qui auront les mêmes clés, il est possible de faire un déni de service)
		\item peu de switch(C) car algo linéaire -> le compilateur lit les cas les uns après les autres
	\end{itemize}
\end{itemize}

Qu'est ce qui empêche un système d'être en temps réel?
\begin{itemize}
	\item Temps CPU
	\item les entrées sorties
	\item les interruptions et exceptions
	\item les zones critiques (zones à verrous, forçant des processus à attendre si déjà accédées)
\end{itemize}

Au démarage, le processeur reçoit une interruption Reset -> l'adresse mémoire du Reset contenue dans le vecteur d'inerruption (liste d'adresse mémoire) est stocké dans le compteur ordinal(registre contenant l'@ de la prochaine instruction à éxécuter). Ce programme est le BIOS (Basic Input Output System). Il démarre un autre programme intermédiaire, généralement très court (GRUB, LILO), qui charge tout le système en mémoire et démarre son exécution.
Le système ((Linux) Noyau + Initrd/Initramfs (système de fichier contenant le nécessaire pour faire démarrer le système)) initialise la mémoire. Puis, il initialise le système de fichiers, la mémoire virtuelle (définie par le Memory Management Unit) et l'ordonanceur. Lancement de init, qui démarre la machine. Historiquement, init regarde un fichier nommé /etc/inittab et /etc/init.d/...  (/etc/rc2.d/... -> lien symboliques vers /etc/init.d/..). Lancement de udev qui gère les périphériques.

reprogrammer le MMU

\begin{enumerate}
	\item On restaure les registres de la tâche que l'on reprend
	\item On quitte l'interruption
	\item le processus récupère le compteur ordinal et le registre de status
	\item la nouvelle tâche reprend son éxécution
\end{enumerate}

\subsection{Ordonnancement}

\[Non Exitant\] -création->  <-destruction- \[Non Opérationnel\] -lancement-> <-Arrêt- \[Opérationnel\]

Tous les processus envoient une valeur à leur père quand ils se terminent. SI le père ne peut recevoir la réponse, ils deviennent des zombies en attendant.

\[en attente\] --Réveil--> \[Prêt\]  --Allocation--> <--préemption-- \[courant\]
     \^						    		    v
    \|-------------------------------blocage------------------------\|

}


\section{Tâches}
Définies par un PID. Toutes les taches ont un père, sauf \textit{init}.\\ Une tâche ne peut écrire dans la mémoire d'une autre tâche -> mémoire séparées\\ 3 zones mémoires :
\begin{itemize}
	\item Code (mémoire exécutable. Le MMU dit si une zone mémoire peut être lue, écrite ou éxécutée)
	\item Pile
	\item données
\end{itemize}

La pile (fifo) est différente du tas (utilisé pour malloc et free, de type lifo).\\ ulimit permet de limiter pour un processus la taille mémoire allouée, le temps d'éxécution, la taille des entrées sorties.\\
Mémoire verrouillée (sticky bit) Si on veut faire du temps réel, on met le bit t pour qu'un processus ne quitte jamais la mémoire centrale :\\ rwxr\_xrr\_x\\rwsr\_sr\_t\\
-> chmod 1755 <fichier>

Imprédicabilité de malloc (aucune idée du temps que ça met)

cannaux de communication : Inter Processus Communication, permettant d'utiliser la mémoire partagée, les verrous, les messages. \\Les IPCs ont une API pas terrible. (ne jamais faire kill -9, car les ressources utilisées ne seront pas libérées).\\

La liste des fichier ouverts : File Descriptor -> stockée dans la table des processus coté kernel.

Droit d'un processus : UID/GID. EUID/EGID -> Droits effectifs

Groupes d'éxécutions : nohup déconnecte l'entrée/la sortie standard/d'erreur. Un fichier peut être détruit, mais continuer à exister s'il n'a pas été fermé.\\

Vue personnelle des systèmes de fichiers : originalement destiné à la racine  : chroot <chemin> définit chemin comme la racine.

Inconvénients des processus : lents à démarrer, consommant des ressources. Solution : utilisation des threads (processus légers). Les threads d'un processus partagent tout, sauf la pile. Ça consome beaucoup moins de ressources. L'inconvénient des threads, c'est qu'il n'y a pas de protection mémoire, et qu'on a besoin de section critique partout. Grosso modo, ne pas utiliser de thread. C'est pas la bonne solution en général. Les threads sont un perte de temps -> maximum deux threads par coeur du processeur.\\

Le multi-tâche n'est pas forcément gagnant.

Ordonnanceur : Hors-ligne(décidé à l'avance)  ou en-ligne\\ Préemptif : l'ordonnanceur peut prendre la main et choisir un processus à éxécuter. Non-préemptif : pas d'interruption d'une tâche en cours.

Type : 
\begin{itemize}
	\item PAPS : premier arrivé, premier servi
	\item tourniquet/round robin : chacun son tour
	\item Priorité : les processus ont une priorité de départ qui évoluent en fonction de leur utilisation.
\end{itemize}

Un ordonnanceur premptif utile des interruptions. Pendant longtemp, la fréquence de l'ordonnanceur était de 100 Hz.

Exclusion mutuelle : \\
c++ \\
\begin{verbatim}
i++
LOAD A
INC 	=> INC A
STORE A
\end{verbatim}

Pour un i++, on ne déclare pas int i, mais volatile int i;\\
Cela dit au compilateur de ne pas garder la variables dans ls registres\\
Il existe une instruction pour bloquer le bus.\\

Comment faire les sections critiques : en mono-processeur, arrêter les interruptions, demander à l'ordonnanceur de ne pas nous suspendre\\
Multiprocesseur : il faut un verrou.\\Le verrou le plus simple : spinlock avec test and set. Cela marche pour des ttes petites sections critiques (minimales en temps cpu).\\Dans le cas long, il faut un dialogue avec l'ordonnanceur (et endormir le processus)\\
Dans Linux depuis quelques années, il existe des verrous spéciaux.

Test and set
\begin{verbatim}
boolean test_and_out(volatile int &valeur){
	if(*valeur)
		return 1;
	*valeur = 1;
	return 0;
}
\end{verbatim}

Baker's algorithm (Leslie Lamport)

Spin Lock :\\
\begin{verbatim}
void spinlock(Spinlock sl){
	while(test_and_out (& sl -> valeur)){
		termine_quantumdetemps();
	}
}

unlock sl -> valeur = 0
\end{verbatim}

Mutex (section longue)=> Il doit gérer l'endormissement et les files d'attentes.\\
La difficulté d'implémentation, très pénible, est qu'on veut être réveillé lorsque le mutex est libéré. Attention : il ne faut pas être réveillé avant se s'endormir


\vskip 3cm
Spinlock : attente active, on attend de pouvoir prendre le verrou.\\

Cache Browsing : 
\begin{enumerate}
	\item Un processus cherche à accéder à une zone mémoire, constituée de [verrou|données] ou un autre processus met à jour les données. La zone de 64 octets mis en cache. 
	\item la vitesse d'éxécution du premier processus va mettre dix fois plus de temps.
\end{enumerate}

\subsection{Mutex vs Semaphore}
Dead lock : On prends deux tâches A et B. La tache B verrouille X(Lock(X))
\begin{verbatim}
Tache A		Tache B
	|		   Lock(X)
	|			|
	Lock(Y)		|
	|		   Lock(Y)
	Lock(X)
\end{verbatim}

Verrouiller et déverrouiller toujours dans le même ordre.
Valgrind a des outils utiles.\\

Inversion de priorité\\
On peut se passer de thread

\vskip 1cm
RCU : Read Copy Update :\\
"petite" structure de donnée

Celui qui modifie :\\
Il alloue une nouvelle zone mémoire\\
Il recopie la structure de données dedans(on stocke la date à laquelle on fait le changement \\
Il la met à jour\\
Libération de la mémoire : tt les threads : ...  : je n'ai plus de référence -> le thead qui libère compare les dates\\

\vskip 1cm
Lecteur/écrivain:\\
Plusieurs lecteurs, un seul rédacteur.\\
\begin{verbatim}
void lock_ecrivain(mutexPC m){
	lock(m->ecrivain_en_attente);
	lock(m -> ecrivain);
	unlock(m->ecrivain_en_attente),
}

void unlock(MutexPC m){
	unlock(m->ecrivain);

void lock_lecteur(MutexPC m){
	lock(m->lecteur);
	lock(m->ecrivain_en_attente);
	m->nb_lecteur++;
	if(m-> nb_lecteur == 1)
		lock(m->ecrivain);
	unlock(m->lecteur)
}

void unlock_lecteur(MutexPC m){
	lock(m->lecteur)
	m->nb_lecteur--
	if(m->nb_lecteur == 0)
		unlock(m->ecrivain);
	unlock(m->lecteur);
	unlock(m->ecrivain_en_attente);
}
\end{verbatim}

\vskip 1cm
Mémoire transcationnelle :\\
\begin{itemize}
	\item Début de transcation
	\item Lire des infos
	\item Ecrire des infos(tout est tracé
	\item Fin de transaction. Si une des donnée lue a été modifiées, on efface tt et on recommence
\end{itemize}

Inconvénients : coute très cher en CPU si fait de manière logicielle : (existe en Hard(CPU) mais la transaction est de taille limitée)\\

\vskip 1cm
Synchronisation :\\
Une tâche envoie un signal pour libérer les autres tâches.\\
\begin{verbatim}
void cmd_wait(Cmd c, Mutex m){
	lock(c->mutex);
	ajouter(c->liste, tache_courante());
	unlock(c->mutex);
	unlock(m);
	sleep();
	lock(m);
}
\end{verbatim}
\vskip 1cm

Signaux :\\
Historiquement A\&T System V\\

Inconvénients : 
\begin{itemize}
	\item pas de file d'attente
	\item pas de priorité
	\item on ne connait pas l'auteur du signal
	\item on ne connait pas le numéro de signal
\end{itemize}

\vskip 1cm
BSD -> sigvec\\

\subsection{Mémoire}

\begin{itemeize}
	\item Sticky bit 
	\item Préallouer la mémoire et la pile.
\end{itemize}

\vskip 1cm
\subsection{Horloge}
Horloge par défaut : quantum de temps.\\
Le RTC est une horloge lente à programmer.\\
Horloge précise autant que possible pour le temps réel.

Watchdog : compteur se décrémentant, avec reboot (reset) lorsque le décompte atteint zéro. L'ordonnanceur réinitialise le compteur si l'ordonnancement à réussi.



\newpage
\section{Ordonnancement Temps Réel}
Type de tâche ordonnancés :
\begin{itemize}
	\item Périodique
	\item Sporadique, temps minimum entre deux déclenchements
	\item Apériodique (on ne sait pas de combien de temps le CPU à besoin pour les éxécuter)
\end{itemize}

\vskip 1cm
Les attributs d'une tâche sont :
\begin{itemize}
	\item La période t : 0, t, 2t, 3t
	\item L'échéance e : elle doit se terminer avant nt + e si elle a commencé en nt
	\item La capacité : temps maximum d'éxécution (somme des temps : d'attente, d'éxécution)
\end{itemize}

La partie compliquée est de définir la capacité. Il est difficile de savoir si on est dans le pire cas. Ce sont des mesures difficiles à faire.\\
\vskip 1cm
Comment savoir si l'ordonnancement est temps réel?\\
Rate Monotonic : ordonnanceur préemptif 
\begin{itemize}
	\item les tâches avec la période la plus courtes sont les plus prioritaires
	\item Échéance = Période
\end{itemize}
$\sum_{i=1}^n \frac{C_i}{P_i} =< n * (2^{\frac{1}{n}} - 1)$ avec $C_i$ la capacité du processeur et $P_i$ la période d'utilisation (soit le taux d'utiliation du temps CPU)\\
Pour $n = 2$ On obtient 0.78. Pour $n -> \infty$ on a 0.69.\\

\vskip 1cm
La seconde méthode est celle de la zone critique. Le pire cas existant dans un système temps réel est quand ttes les tâches démarrent en même temps. Si, dans ce pire cas, ttes les tâches respectent leur première échéance, elles respecteront les suivantes.\\
$W_i(t)$ calcule le temps ou se terminera la tâche $i$ si elle n'est pas interrompue. $W_i(t) = \sum_{j =0}^{i-1} C_i \frac{t}{p_j} + C_i$.\\
Pour la tâche la plus prioritaire :\\$W_1(0) = C_1$\\
$W_1(C_1) = C_2$\\
$W_2(C_1) = C_1\frac{C_1}{P_1} + C_2 =  C_1 + C_2$\\
$W_2(C_1 + C_2) = C_1\frac{C_1 + C_2}{P_1} + C_2 = $

Ceci permet de prouver que des systèmes sont temps réels. La grosse imprécision vient du calcul du temps des tâches.


\vskip 1cm
\subsection{Dead Line Monotonic}
$E_i = \delta P_i$\\
Temps réel si :\\
$\delta =< 0.5 $ Taux occupation $< \delta$\\


Ordonnancement hors ligne : ordonnancement cyclique(non préemptif)\\
Cycle majeur : ordonnancement précalculé -> la machine passe son temps à faire en boucle des cycles majeurs. Chaque cycle majeur contient des cycles mineurs déclenché par une interruption.

Cycle majeur : multiples du PPCM des périodes des tâches\\
cycles mineurs : plus petites périodes.

Avantages :
\begin{itemize}
	\item Pas de secton critique
	\item rien d'aléatoire : tt se reproduit de la même façon
\end{itemize}

\vskip 1cm
\subsection{ordonnanceur Temps Réel en ligne}
Algorithmes :
\begin{itemize}
	\item HPF : High Priority First -> on laisse la main aux tâches les plus prioritaires, et uniquement à elles. Ne peut pas forcément toujours ordonnacer en tps réel\\
	\item EDF : Earliest DeadLine First -> On laisse la main au processus qui se termine le plus tôt. Cet algorithme est forcément meilleur que HPF, car moins de changement de contexte que le HPF.
	\item LLF : Least Laxity First -> Celui qui a le moins de marche de manoeuvre est prioritaire. Meilleur que EDF. L'inconvénient de LLF pour le programmeur est la difficulté à évaluer la capacité.
\end{itemize}
En cas de surcharge : 
\begin{itemize}
	\item HPF : on sait ce qu'il fait
	\item EDF : fait n'importe quoi
	\item LLF: on sait au démarrage qu'il y a undiffhist = difference(hist, frame); problème. On peut lancer une version light de la tâche
\end{itemize}

\vskip 1cm
Cas des tâches non périodiques :\\
idée : une tâche qui est à faible priorité et qui contient un ordonnanceur "normal". Elle est périodique et n'a pas d'échéance.\\
Problème : latence de démarrage. \\
Solution : Pour résoudre ce problème, on dit que c'est la tâche la plus prioritaire. On calcule une capacité maximum pour que le système soit temps réel. A la fin de la capacité, on (l/b)aisse la priorité
\end{document}
