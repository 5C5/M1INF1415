\documentclass[11pt]{article}
\usepackage[utf8]{inputenc}
\usepackage[T1]{fontenc}
\usepackage[frenchb]{babel}
\usepackage{ansmath}

\title{Analyse d'image}

\begin{document}

\maketitle

\section{Outils Fondamentaux}
\subsection{définitions}
L'image peut-être plongé dans le plan dscret $Z^2$
Point discret : points de $Z^2$, correspondant au pixel centré en ce point.
Distances spatiales utilisées :

\begin{itemize}
	\item $D_2 (P,Q) = ((x_P - x_Q)^2 + (y_P - y_Q)^2)^{1/2}$
	\item $L_1 (P, Q) = |x_P - x_Q| + |y_P - y_Q|$
	\item $l_\infty  (P, Q) = MAX(|x_P - x_Q|, |y_P - y_Q|)$
\end{itemize}

Deux points discrets P et Q sont dits 4-adjacents ssi $L_1(P, Q) = 1$.\\
$|x_p - x_q| + |y_p - y_q| = 1$\\

Deux points discrets sont dit 8-adjacents ssi $L_\infty (P, Q) = 1$.\\
$Max(|x_p - x_q|,|y_p - y_q|) = 1$

Voisinage :
\begin{itemize}
\item V : voisinage d'un pixel P au sens d'une distance D
\item $V(P) = P' tel que  D(P, P') =< \epsilon}$. $\epsilon$ valeur donnée. En général $\epsilon = 1$.
\item Si $\epsilon = 1$, $V_4$ <=> 4-connexité
\item $V^4(P) = {P', P' \in I L_1(P, P') = d_4(P, P') =< 1}$. P $\in$ voisinage. C'est la connexité simple : il existe un recouvrement (aussi fin qu'il soit). La connexité par arc consiste en l'existence d'un arc permettant de passer de n'importe quel point à un autre.
\end{itemize}

\vskip 2cm
Chemin discret : un chemin discret k-connex est une suite de points discrets ($P_0$, $P_1$, ..., $P_n$), $P_{i-1}$ et $P_i$ sont \texbf{k-adjacents}.\\

Code de Freeman : la suite ($P_0$, $P_1$, ..., $P_n$) est représentée par ($P_0$, $d_0$, ..., $d_{n-1}).$ La valeur $d_i$ code le déplacement relatif de $P_i$ à $P_{i+1}$, avec cadrant(droite 0, haut 1, gauche 2, bas 1) ou octant(0 à 7)\\

Ensemble k-connexe : ensemble de points discret E tels que quelque soit P et Q ∈ E, il existe un chemin discret k-connexes dans E d'extrémités P et Q). La composante connexe d'un ensemble de points discrets est l'ensemble connexe maximal (ou classe d'équivalence pour la relation d'adjacence) E tels que ....
\vskip 1cm

La convolution : \\
Soit h : RxR -> R;\\
Soit f une fonction donnée; f : RxR -> R;\\
La convolution de f par h est définie : \\
$(f * h)(x, y) = \int \int_R f(u, v)h(x-u, y - v) dudv$\\
$(h * f)(x, y) = \int \int_R h(u,v)f(x - u, y -v ) du dv$\\
$f * h(x, y) = h * f(x, y)$

\vskip 2 cm
Dans le cas discret :\\
Soit I le support d'une image, \\
h : [$m_1$, $m_2$] x[$n_1$, $n_2$] -> R\\

La convolution de I par H est définie par :\\
$(I * h)(x, y) = \sum^{u=m_2}_{u=m_1}\sum^{v=n_2}_{v=n_1} h(u, v)I(x- u, y-v)$

h est appelé noyau de la convolution. On utilise également le terme de filtre ou masque.\\


exemple :
$h(u, v) = \frac{1}{5}\mat { 0 & 1 & 0\\ 1 & 1 & 1 \\ 0 & 1 & 0}$\\
$I(x_0, y_0) = \frac{1}{5}(
I(x_0 -1, y_0) + I(x_0, y_0+1) + I(x_0, y_0) + I(x_0 +1, y_0) + I(x_0, y_0 -1))$

Dans le cas continu, h est continue {+général que la continuité}

support de $h : [-l, l] : h(x) = 0$ si $x \in [-l, l]$

\vskip 1cm

Si on est dans le domaine discret :\\
si h et a support fini -> matrice finie(l'évaluation numérique de h): matrice centrée sur un point'$x_0$, $y_0$.\\
On ne peut pas pas appliquer la convolution sur les points des bords : pas de voisinnage.


Quelques propriétés de la convolution :
\begin{itemize}
	\item associativité : $f * h * g =(f*h) *g$
	\item commutativité : $f* h = h * f$
	\item dist de +/*
	\item dérivabilité : $(f *h)' = f * h' = f' * h$
\end{itemize}

\newpage

\section{Traitements à base d'histogramme}

Histogramme : outil de base pour l'étude des capteurs ou de la dynamique d'une scène, il est utilisé par divers opérateurs d'analyse. Permet de voir comment la charge, la dynamique se réparti dans une scène.\\

La manipulation d'un histogramme permmet de changer l'apparence d'une image, de catégoriser des éléments.\\

L'histogramme d'une image réprésente la répartition des pixels en fonctio de leurs valeurs. Il fournit diverses informations (statistiques, , d'ordre, la répartition de la dynamique).\\

H(X) donne le nbre d'élément dont la valeur de la caractéristique est égale à X

\vskip 2cm

\section{}

Egalisation d'histogrammes 

\vskip 1cm

Spécification d'histogramme : rendre la distribution d'intensité d'un distribution spécifiée à l'avance\\
conservation des positions et sens des transistions avec une transformation F -> G croissante\\
Distribution de référence = image ou région d'une image\\

Filtrage spatial : lissage et réduction du bruit : transformation basée sur le voisinage (spatial) d'un point (x, y) sur la base d'opérateurs et de convolutions.\\
On évalue et on modifie la valeur d'un point en fonction de celles de ses voisins : $F(x_0, y_0) = f * g(x_0, y_0)$ avec g le masque/noyau : répartition de valeurs de points. Il faut veiller à ce que la somme des valeurs des différents points soit égale à 1.\\

Filtrage par convolution : multiplication dans le domaine fréquentiel -> convolution dans le domaine spatial\\

Filtre de lissage obtient noyau de convolution symétrique et normalisé\\
somme des coef = 1.


\vskip 0.5cm
Moyenne : $h(x, y) = \frac{1}{M}$ ou $h(x, y) = \frac{1}{\lambda^2}$

M : nbre de pts du voisinage, ou $\lambda * \lambda$.
\vslip 0.5cm
Gaussienne : $h(x, y) = \frac{1}{2\pi\sigma^2}exp(\frac{-x^2 - y^2}{2\sigma})$

\vskip 0.5cm
Exponentielle : $h(x, y) = \frac{\gamma^2}{4}\exp(-\gamma(|x| + |y|))$\\

\newpage
\section{chapitre 2 : quelques méthodes}
\subsection{segmentation}
Def : partitionner l'image en zone homgènes selon un critère déterminé : couleur, texture, niveau de gris, indice...

Il faut que l'union de ts les éléments retrouvés soit le recouvrement total de l'image. Deux zones contigues n'ont en commun que la frontière.\\
Frontière d'une région 2D : La frontière d'une région 8-connexes (respectivement 4-connexes) R est l'ensemble des points de R dont au moins un des 4-voisins (respectivement 8-voisins) n'est pas élément de R => la frontière est composée de chemin 8-connexes(respectivement 4-connexes).\\

global <---> local : le cerveau humain passe de l'un a l'autre sans arrêt.\\

Plusieurs approches :
\begin{itemize}
	\item Approches glbales : on voit globalement comment découper l'image, via les statistiques (histogramme). \\Avantage : méthode très rapides\\
		Inconvénients : Elles ignorent les informations de proximité
	\item approches locales : (relaxation, méthodes markoviennes)
\end{itemize}

Méthodes itératives
\begin{enumerate}
	\item Initialisation
	\item R. Production
	\item Condition d'arrêt
\end{enumerate}

Région growing : difficulté d'estimer le nbre de région => initialisation. Méthode classique : 
\begin{enumerate}
	\item on sème des germes
	\item on procède à l'agglomération/accroissement des points autour du germe(pixel(i, j) -> ensemble de pixels $R_i$ -> pts frontières). 
	\item Un pt frontière devient germe. On obtient un ensemble d'îlot => Une région = plusieurs ilôts
	\item fusion de sous-région (postulat : une région est maximale). On fusionne les régions adjacentes et semblables.\\
		Csque (pré-requis )
\end{enumerate}

stratégie de pose de germes: pose aléatoire, en découpant l'image de façon équilibrée.
Le critère d'arrêt de l'agglomération des points aux germes (frontière).\\
un pt frontière est un point final ou point terminal ssi\\
$\forall p \in $frontière R (aucun pt ne peut être ajouté)\\
Ou deux régions se rencontrent.

\vskip 1cm 

2 Problèmes :
\begin{itemize}
	\item L'ensemble obtenu ne constitue pas un recouvrement, il restera des points non affectés à des groupes (comp élémentaire)
	\item l'existence de compo élémentaire adjecents - semblables -> fusion de comp éléments adjacents
\end{itemize}


\vskip 1cm
Pose des germes : Stratégie
\begin{itemize}
	\item Aléatoire
	\item Régulire( grille) -> aléatoire supervisée
\end{itemize}

\vskip 1 cm
Stratégie de la propagation : 2 façons
\begin{itemize}
	\item Propagation en profondeur : dès que l'on a un pixel, on suit sa propagation, on part de ce pixel, jusqu'a ce qu'on ne puisse plus propager
	\item Propagation en largeur : par couronne
\end{itemize}


Région Growing :
\begin{itemize} 
	\item Avantage : rapide, conceptuellement très simple
	\item inconvénients : c'est une méthode locale(pas de vision globale du problème), il faut tenir compte de l'homogénéité globale, algo très senesible au choix du mode de propagation, il faut faire attention au choix des structures de données.
\end{itemize}

\subsection{Split And Merge}
On prend l'image, on la découpe autant qu'on peut, puis on remet les éléments ensemble.

Étapes :
\begin{itemize}
	\item On subdivise l'image en zones homogènes élémentaires (C1) -> Split
	\item on fusionne les zones adjacentes (C2) -> Merge
\end{itemize}
Pour que la subdvision soit automatique, on utilise une structure de données spécifique et un critère particulier. Il est possible d'utiliser un arbre, mais dans ce cas, la gestion de la fusion sera compliqué (les zones adjacentes ne seront pas forcément voisines dans l'arbre). Il faut le combiner avec un graphe d'adjacence.\\

\vskip 1cm
Quelques remarques : 
\begin{itemize}
	\item Difficulté  : structuration et codage, Gestion des adjacences
	\item critères : division et fusion
	\item Finesse de la subdivision (compromis : finesse <-> temps de calcul)
\end{itemize}

La segmentation par ligne de partage des eaux : famille des méthodes de segmentation d'mage qui consièdrent une image à niveaux de gris comme u relief topographique, dont on simule l'inondation : -> on remplit les creux de l'histogramme (Water Shade) jusqu'au débordement.


\vskip 1cm
\begin{enumerate}
	\item Décision déterministe
	\item Hypothèse -> probabiliste\\
		->confirmer ensuite
\end{enumerate}
stochastique : Gibbs, Markov.

\vskip 1cm
A : ensemble des pixels (i, j)\\ 
attribut : niveau de gris (valeur de couleur).\\
Hypothèse on dispose de N classes(région). Assignation /affect des pixels dans les différentes classes/régions\\

A un pixel(i, j) on associe un vecteur de probabilité de taille N\\
$(i, j) -> P(i, j) = [P(i, j,1); P(i, j, 2); ...; P(i, j, N)]$\\
$P(i, j, l)$ : probabilité pour (i, j) d'être dans la classe/région l.\\
But : de faire évoluer chaque vecteur $ P(i, j) \approx \alpha$ -> itérative\\
$0 << P(i, j, l) << 1$ soit $ \sum_l P(i, j, l) = 1$

\vskip 1cm 
Règles de production/ Mise à jour (Rosenfeld)
On donne une quotation négative/très faible à certains éléments pour pondérer leur probabilité en fonction de leur cohérence : on tient compte de la connaissance que l'on a.\\
Rosenfeld propose de soit calculer la corrélation entre ttes les classes, soit de calculer une estimation

$r((i, j) \in l, (u, v) \in k)$ compatibilité $\in [-1, 1]$\\


\section{}
\begin{itemize}
	\item Relaxation probabiliste
	\item méthode Markovienne (processus de Markov) : Le présent dépend du passé, et l'état à un temps $t_i$
\end{itemize}
Cela correspond à l'assignation d'un pixel ij à une classe.

\vskip 1cm
Plusieurs étapes :
\begin{enumerate}
	\item Répartition initiales : chaque pixel (i, j) est affecté à une classe donnée avec un indice de certitude (probabilité)\\
		-> Contexte :\\
			-> quantifié par des compatibilités (....), de configurations :\\
			-> classe c1/ a la classe C2\\
\end{enumerate}

On dispose des probabilités (i, j) $\in R_l -> P^{(n)}_{ij} {\\
	P^{(n)}_{(ij)}(R1)\\
	P^{(n)}_{(ij)}(R2)\\
P^{(n)}_{(ij)}(RL)\\}$


$cmp((i, j) \in R_l, (u, v) \in R_m)$\\
$cmp(R_l, R_m)$\\

$\sum_{(u, v)} \sum_l cmp(R_l, R_m) * P'uv(R_l)$


\vskip 2cm

\end{document}
