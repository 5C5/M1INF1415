\begin{document}
Correction du Cylindre : \\
n points sur le cercle du haut, n pts sur le cercle du bas : 2n + 1 pts\\
Indice de a : 2n, indice de b : 2n+ 1\\

Utilisation d'une base orthonormé i, j et k, avec :\\
$k = b-a/||b - a||$\\
$i = \frac{k}{(orthogonal à k)}$\\
$j = k ∧ i$

$P_{i \in [0, n-1]} = a + r(cos\frac{2\pi i}{n} i + sin(\frac{2\pi i}{n}j))$\\
et ainsi de suite pour remplir la geométrie.\\

La topologie :\\
$i \in[0, n-1]$ la face inférieure : [$(i+1)\%n$|i | 2n]\\
$i \in[0, n-1]$ la face supérieure : [n + i| $n +(i+1)\%n$| 2n +1]\\

Pour les faces latérales :\\
[i| $(i+1)\%n$| $n + (i+1)\%n$]\\
[i| $n+ (i+1)\%n$ | $n + i$]\\


Remarque :\\
On a 4n Triangles et 2n +2 sommets. Si n $\approx$ 10 -> 40T\\
et si n $\approx$ 20 -> 8OT\\
Bien approximer les arrondis coute forcément cher!\\

Modélisation :
\begin{itemize}
	\item Low Poly : nbre de triangle faible : domaine du temps réel extrêment contraint -> branche du domaine artistique consistant à représenter un obj avec le minimum de triangle -> Permet d'augmenter le nbre d'objet
	\item High Poly
\end{itemize}

Remarque : \\
La sphère est un bon exemple : 2n échantillons pour l'équateur, et 2n pour un méridien :\\
$2n^2$ Sommets, et $\approx $ O($n^2$).

TODO : \\
1 page pdf avec shoot des maillages + et nom des f° membres associées. + les statistiques

Exercice d'approfondissement : \\
comment créer cette forme -> courbe $\gamma$, -> objet de révolution (extrusion). Pour simplifier, la courbe $\gamma$ équation polynomiale de degré faible (cubique spline) -> repère de Fresnet.

Exercice d'approfondissement 2\\
écrire un pro qui crée 1 temple grec à partir de deux points et d'une hauteur

google Shader Toy


\newpage
\vskip 3cm
TP2 : \\
->un milestone (rendu intermédiaire) dans 15jours
->un rendu dans trois semaines

Le prof file : code.txt -> shadertoy.

regarder fonctions de bruits, et leur codage (connaissance) 1-2h. Écrire des fonctons de texture : -bois (bois tordu), granite, métal rouillé, lave, métal brossé.
Coder au minimum quatre matériaux.




TODO : coder l'érosion th (facultatif) ~ 4- 6h

TODO : TP#4 coder n distribution de végétation et les comparer.
Utiliser des arbres symboliques. f(env).

TODO : coder la simulation d'ecosysteme.
Générer 1 image 2D vue de haut.
\end{document}
