\documentclass[11pt]{cours}
\usepackage[utf8]{inputenc}
\usepackage[T1]{fontenc}
\usepackage[frenchb]{babel}


\title{Synthèse d'image}

\begin{document}
\maketitle

\section{Modèle}
Modélisation : tte méthode permettant de réprésenter un objet dans le cadre 3D et de le manipuler.
Un modèle, c'est :

\begin{itemize}
	\item Un ensemble de données, en deux type
	\begin{itemize}
		\item soit des données pour décrire la surface (2D). Les modèles surfaciques sont adaptés à la visualisation temps réel, car les cartes graphiques actuelles en sont capables
			\item soit des données pour décrire le volume (3D).
	\end{itemize}
	\item les algo de manipulation qui vont avec : 
	\begin{itemize}
		\item Les requêtes élémentaires : ($pt \in O$?, calculer l'intersection entre une droite $\delta$ et un objet $\sigma$, ...
		\item Les ensembles : l'intersection entre deux objets existe-t-elle?, déformation, animation...
		\item conversion. Comme certaines opérations sont plus faciles avec certains modèles et beaucoup plus dur avec d'autres, il est nécessaire de pouvoir convertir les données d'un modèle à un autre.
	\end{itemize}
\end{itemize}

\section{modèle surfacique}

\begin{itemize}
	
	\item Surfaces paramétrés : $S = {p(u, v)  \quad (u, v) \in \omega \include RxR \\ p(u, v) : \omega  -> R^3\\ (u, v) -> p(u, v)$\\ On travaille avec des pts de contrôle\\
	exemples: 
	\begin{itemize}
		\item $p(u, v) = \mat{ u \\ v \\ 0}$
	\end{itemize}
	Souvent, on va travailler avec des $\omega$ simples $[O, 1]^2$ et p(u, v) simple ($R_n[u, v]$\\ On définit des morceaux (patchs) de surfaces 3D. \\ Utilisation de polynômes\\
	Famille de patches: Bézier, Splines, Coons...\\
	Ce sont des surfaces ayant de bonnes propriétés (polynomiales, donc contrôles de beaucoup de propriétes). Mais le problème vient du raccord entre plusieurs morceaux (préservation de continuité -> contraintes venant de quatres sources à satisfaire -> le degré du polynôme augmente pour absorber les contraintes -> on peut passer de degré 3 et 4 à des degrés 24.\\
	Conversion d'une surface paramétré en un ensemble de triangle (maillage).
	
	\item Surfaces de subdivision : -> Aller lire ce que sont les surfaces de subdivision -> ultra utilisée dans l'industrie graphique, notemment animation. Pas de polynome, mais analyse spectrale. Passage d'un contrôle grossier à une surface lisse par un processus de subdivision. Gueri's Game Youtube. Subdivision : chaque triangle, à chaque itération, est coupé en quatre triangle, avec un raffinement récursif.

	\item Maillaige : Un maillage est constituée de deux types de données :
		\begin{itemize}
			\item des données géométriques, qui permettent le déplacement
			\item des données topologiques (sommets V, arêtes E, faces F, liés les un aux autres). Elles permettent de savoir qui est connecté à qui, ce qui est utile quand on change de topologie, c-a-d quand on change le nbre d'arete, de sommet ou de face. 
		\end{itemize}
		Modèle eulérien  : Relation d'Euler = $V+ F - E = 2$\\ Les faces peuvent être n'importent quel type (triangle, carré, etc).\\
		Structure minimale pour représenter un maillaige : 
		\begin{itemize}
			\item Une structure pour stocker la géométrie : un tableau (QVector<Vector>)
			\item Une structure pour la topo minimale : un tableau (QVector<int> tcom) pour stocker des triplets, contenant les indices des points dans le tableau de la géométrie.
		\end{itemize}
		$\forall$ opération de mvmt solide (trslat°, rotat°, ..) => géom parcours\\
		L'affichage est un parcours de la topologie (avec un parcours de la géométrie indexée par la topologie)\\
		Lorsqu'il y a une déformation de maillage (mvmt, par exemple), on surmaille (on rajoute un nmbre de pts) à l'avance la zone destinée à une animation non solide.\\
		Pour gérer des maillages aux éléments vairables, on stocke la somme du nbre de point par mesh dans un autre tableau.\\
		On a souvent besoin d'info supplémentaire : la normale au sommet (pour chaque sommet), des textures, ...
		Pour cela, soit on fait des tableaux supplémentaires de données, soit on entrelace les données.\\ Souvent, on aura besoin de normales au sommet d'un maillage. La normale à un triangle abc : $ n = \frac{(b-a)\^(c-a)}{(b-a)\^(c-a)}$(on redivise par lui-même pour avoir un vecteur unitaire). tous les calculs d'illuminations prenent en compte la normale aux facettes.\\
		
\end{itemize}

\end{document}
