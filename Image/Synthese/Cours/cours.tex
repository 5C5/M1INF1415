\documentclass[11pt]{article}
\usepackage[utf8]{inputenc}
\usepackage[T1]{fontenc}
\usepackage[frenchb]{babel}


\title{Synthèse d'image}

\begin{document}
\maketitle

\section{Modèle}
Modélisation : tte méthode permettant de réprésenter un objet dans le cadre 3D et de le manipuler.
Un modèle, c'est :

\begin{itemize}
	\item Un ensemble de données, en deux type
	\begin{itemize}
		\item soit des données pour décrire la surface (2D). Les modèles surfaciques sont adaptés à la visualisation temps réel, car les cartes graphiques actuelles en sont capables
			\item soit des données pour décrire le volume (3D).
	\end{itemize}
	\item les algo de manipulation qui vont avec : 
	\begin{itemize}
		\item Les requêtes élémentaires : ($pt \in O$?, calculer l'intersection entre une droite $\delta$ et un objet $\sigma$, ...
		\item Les ensembles : l'intersection entre deux objets existe-t-elle?, déformation, animation...
		\item conversion. Comme certaines opérations sont plus faciles avec certains modèles et beaucoup plus dur avec d'autres, il est nécessaire de pouvoir convertir les données d'un modèle à un autre.
	\end{itemize}
\end{itemize}

\section{modèle surfacique}

\begin{itemize}
	
	\item Surfaces paramétrés : $S = {p(u, v)  \quad (u, v) \in \omega \include RxR \\ p(u, v) : \omega  -> R^3\\ (u, v) -> p(u, v)$\\ On travaille avec des pts de contrôle\\
	exemples: 
	\begin{itemize}
		\item $p(u, v) = \mat{ u \\ v \\ 0}$
	\end{itemize}
	Souvent, on va travailler avec des $\omega$ simples $[O, 1]^2$ et p(u, v) simple ($R_n[u, v]$\\ On définit des morceaux (patchs) de surfaces 3D. \\ Utilisation de polynômes\\
	Famille de patches: Bézier, Splines, Coons...\\
	Ce sont des surfaces ayant de bonnes propriétés (polynomiales, donc contrôles de beaucoup de propriétes). Mais le problème vient du raccord entre plusieurs morceaux (préservation de continuité -> contraintes venant de quatres sources à satisfaire -> le degré du polynôme augmente pour absorber les contraintes -> on peut passer de degré 3 et 4 à des degrés 24.\\
	Conversion d'une surface paramétré en un ensemble de triangle (maillage).
	
	\item Surfaces de subdivision : -> Aller lire ce que sont les surfaces de subdivision -> ultra utilisée dans l'industrie graphique, notemment animation. Pas de polynome, mais analyse spectrale. Passage d'un contrôle grossier à une surface lisse par un processus de subdivision. Gueri's Game Youtube. Subdivision : chaque triangle, à chaque itération, est coupé en quatre triangle, avec un raffinement récursif.

	\item Maillaige : Un maillage est constituée de deux types de données :
		\begin{itemize}
			\item des données géométriques, qui permettent le déplacement
			\item des données topologiques (sommets V, arêtes E, faces F, liés les un aux autres). Elles permettent de savoir qui est connecté à qui, ce qui est utile quand on change de topologie, c-a-d quand on change le nbre d'arete, de sommet ou de face. 
		\end{itemize}
		Modèle eulérien  : Relation d'Euler = $V+ F - E = 2$\\ Les faces peuvent être n'importent quel type (triangle, carré, etc).\\
		Structure minimale pour représenter un maillaige : 
		\begin{itemize}
			\item Une structure pour stocker la géométrie : un tableau (QVector<Vector>)
			\item Une structure pour la topo minimale : un tableau (QVector<int> tcom) pour stocker des triplets, contenant les indices des points dans le tableau de la géométrie.
		\end{itemize}
		$\forall$ opération de mvmt solide (trslat°, rotat°, ..) => géom parcours\\
		L'affichage est un parcours de la topologie (avec un parcours de la géométrie indexée par la topologie)\\
		Lorsqu'il y a une déformation de maillage (mvmt, par exemple), on surmaille (on rajoute un nmbre de pts) à l'avance la zone destinée à une animation non solide.\\
		Pour gérer des maillages aux éléments vairables, on stocke la somme du nbre de point par mesh dans un autre tableau.\\
		On a souvent besoin d'info supplémentaire : la normale au sommet (pour chaque sommet), des textures, ...
		Pour cela, soit on fait des tableaux supplémentaires de données, soit on entrelace les données.\\ Souvent, on aura besoin de normales au sommet d'un maillage. La normale à un triangle abc : $ n = \frac{(b-a)\^(c-a)}{(b-a)\^(c-a)}$(on redivise par lui-même pour avoir un vecteur unitaire). tous les calculs d'illuminations prenent en compte la normale aux facettes.\\
		
\end{itemize}


\newpage
\section{Modèle surfacique}
Modèles qui s'int au volummes, in/out (bool), type de matière(int).\\
Exemple de modèle volumique :
\begin{itemize}
	\item Enumeration spatiales (voxels)
	\item Constructive Solide Geometry
	\item surfaces implictes -> nécessites beaucoup de boulot.
\end{itemize}


\subsection{Voxel}
Un voxel est un cube 3D (pixel) qui contient des informations (bool in/out, int type\_mat, ...).\\
Le modèle voxel, c'est du Lego (je cite).\\

selon un abus de langage, un voxel, c'est une grille 3D (souvent régulière) : {une box 3D, int n(subdivision régulière de la boite), un tableau de données de taille $n^3$ représentant}.\\

Remarque : tableau 1D pour stocker 3D : index = $i+ nj +n^2k$. $i= index \%n$. j =$((index - i)(n)\%n)$.

Inconvénients : 
\begin{itemize}
	\item cout possiblement très élevé (O$(n^3)$).
	\item données discrètes -> besoin de lisser
\end{itemize}

Avantages :
\begin{itemize}
	\item in/out immédiat
	\item accès rapides aux datas
	\item structure de données très pratiques en simulation
\end{itemize}


Soit $p \in R^3$, comment savoir dans quel cube voxel je suis?\\
$\mat{u \\ v \\ w} = \mat{\frac{p_x - a_x}{b_x - a_x}\\ ....\\ ...} \in [0, 1]^3$\\
$\matrix{i\\j\\k} = \matrix{E(u x n)\\ ...\\...}$.

remarque : optimisation technique
index $= i + nj + n^2k$ peut poser des problèmes d'optimisation. Car accèder à des voxels cote à cote selon i se fait avec un + 1 dans la mémoire, mais accéders à des vocels cote a cote selon j ou k se fait en + n. Et la mémoire n'aime pas. Donc il existe des techniques plus avancées pour gérer ce problème.

-> E.Haine & T.Möller, \textit{Real time Rendering}, 3 eme édition, AK Peters.\\
\textit{Fundamental of CG}, P. Shirley, A.K Petes.


\subsection{CSG}
définir un objet par un arbre de construction:
\begin{itemize}
	\item Noeuds : opérateur booléen (Union, intersection, différence)
	\item Feuilles : primitives géométriques (celles pour lesquelles on peut determiner si on est dedans (volumme))
\end{itemize}

exemple : si on fait l'union d'un cylindre et d'un cube, on obtient une masse, et si on y enlève un cylindre couché, on obtient un marteau.\\

On fait du complexe par assemblage simple.

Requête fondamentale sur un mdoèle volumique : bool inside(pt) -> est-ce qu'un pt est à l'intérieur de mon modèle?

exemple d'implémentation : \\
\begin{verbatim}
N::inside(p) = 0
-> Sphère::inside(p)
-> U::inside(p)
\end{verbatim}
Tout noeud doit savoir si un point est à l'intérieur :
\begin{verbatim}
class N {
	public : bool inside(v) const = 0;
}
class S : public N{
	protected :
		V c;
		d r;
	public :
		bool Inside(v) const;
}
class U : public N {
	protected :
		N *a;	//Data sous arbre
		N *b;
	public :
		bool Inside(v) const;
}

bool S::Inside(p) const {
	return (p-c)(p-c) < r?true:false;
}

bool U :: Inside(p) const {
	return a->inside(P) || b->Inside(p);
}
\end{verbatim}

\begin{itemize}
	\item Cube : six tests
	\item Sphère d(p, c)
	\item Cylindre : d(p, $\delta$)
	\item Tore : d(p, $C$)
	\item Cone : complexe
\end{itemize}

Tester si je suis à l'intérieur d'un maillage est extrêment compliqué. Il faut que le maillage soit fermé, cohérent, et 2-variété (Manifold). On utilise le théorème de Jordan, en tirant une demi-droite partant du point. Si le nbre d'intersection avec le maillage est pair, le point est en dehors, sinon, il est dedans. Maillage::Inside(p) -> Triangle::Inside(p).

\end{document}
