\documentclass[11pt]{article}
\usepackage[utf8]{inputenc}
\usepackage[T1]{fontenc}
\usepackage[frenchb]{babel}


\def\R{\textrm{I\kern-0.21emR}}

\title{Synthèse d'image}

\begin{document}
\maketitle

\section{Modèle}
Modélisation : tte méthode permettant de réprésenter un objet dans le cadre 3D et de le manipuler.
Un modèle, c'est :

\begin{itemize}
	\item Un ensemble de données, en deux type
	\begin{itemize}
		\item soit des données pour décrire la surface (2D). Les modèles surfaciques sont adaptés à la visualisation temps réel, car les cartes graphiques actuelles en sont capables
			\item soit des données pour décrire le volume (3D).
	\end{itemize}
	\item les algo de manipulation qui vont avec : 
	\begin{itemize}
		\item Les requêtes élémentaires : ($pt \in O$?, calculer l'intersection entre une droite $\delta$ et un objet $\sigma$, ...
		\item Les ensembles : l'intersection entre deux objets existe-t-elle?, déformation, animation...
		\item conversion. Comme certaines opérations sont plus faciles avec certains modèles et beaucoup plus dur avec d'autres, il est nécessaire de pouvoir convertir les données d'un modèle à un autre.
	\end{itemize}
\end{itemize}

\section{modèle surfacique}

\begin{itemize}
	
	\item Surfaces paramétrés : $S = {p(u, v)  \quad (u, v) \in \omega \include RxR \\ p(u, v) : \omega  -> R^3\\ (u, v) -> p(u, v)$\\ On travaille avec des pts de contrôle\\
	exemples: 
	\begin{itemize}
		\item $p(u, v) = \mat{ u \\ v \\ 0}$
	\end{itemize}
	Souvent, on va travailler avec des $\omega$ simples $[O, 1]^2$ et p(u, v) simple ($R_n[u, v]$\\ On définit des morceaux (patchs) de surfaces 3D. \\ Utilisation de polynômes\\
	Famille de patches: Bézier, Splines, Coons...\\
	Ce sont des surfaces ayant de bonnes propriétés (polynomiales, donc contrôles de beaucoup de propriétes). Mais le problème vient du raccord entre plusieurs morceaux (préservation de continuité -> contraintes venant de quatres sources à satisfaire -> le degré du polynôme augmente pour absorber les contraintes -> on peut passer de degré 3 et 4 à des degrés 24.\\
	Conversion d'une surface paramétré en un ensemble de triangle (maillage).
	
	\item Surfaces de subdivision : -> Aller lire ce que sont les surfaces de subdivision -> ultra utilisée dans l'industrie graphique, notemment animation. Pas de polynome, mais analyse spectrale. Passage d'un contrôle grossier à une surface lisse par un processus de subdivision. Gueri's Game Youtube. Subdivision : chaque triangle, à chaque itération, est coupé en quatre triangle, avec un raffinement récursif.

	\item Maillaige : Un maillage est constituée de deux types de données :
		\begin{itemize}
			\item des données géométriques, qui permettent le déplacement
			\item des données topologiques (sommets V, arêtes E, faces F, liés les un aux autres). Elles permettent de savoir qui est connecté à qui, ce qui est utile quand on change de topologie, c-a-d quand on change le nbre d'arete, de sommet ou de face. 
		\end{itemize}
		Modèle eulérien  : Relation d'Euler = $V+ F - E = 2$\\ Les faces peuvent être n'importent quel type (triangle, carré, etc).\\
		Structure minimale pour représenter un maillaige : 
		\begin{itemize}
			\item Une structure pour stocker la géométrie : un tableau (QVector<Vector>)
			\item Une structure pour la topo minimale : un tableau (QVector<int> tcom) pour stocker des triplets, contenant les indices des points dans le tableau de la géométrie.
		\end{itemize}
		$\forall$ opération de mvmt solide (trslat°, rotat°, ..) => géom parcours\\
		L'affichage est un parcours de la topologie (avec un parcours de la géométrie indexée par la topologie)\\
		Lorsqu'il y a une déformation de maillage (mvmt, par exemple), on surmaille (on rajoute un nmbre de pts) à l'avance la zone destinée à une animation non solide.\\
		Pour gérer des maillages aux éléments vairables, on stocke la somme du nbre de point par mesh dans un autre tableau.\\
		On a souvent besoin d'info supplémentaire : la normale au sommet (pour chaque sommet), des textures, ...
		Pour cela, soit on fait des tableaux supplémentaires de données, soit on entrelace les données.\\ Souvent, on aura besoin de normales au sommet d'un maillage. La normale à un triangle abc : $ n = \frac{(b-a)\^(c-a)}{(b-a)\^(c-a)}$(on redivise par lui-même pour avoir un vecteur unitaire). tous les calculs d'illuminations prenent en compte la normale aux facettes.\\
		
\end{itemize}


\newpage
\section{Modèle surfacique}
Modèles qui s'int au volummes, in/out (bool), type de matière(int).\\
Exemple de modèle volumique :
\begin{itemize}
	\item Enumeration spatiales (voxels)
	\item Constructive Solide Geometry
	\item surfaces implictes -> nécessites beaucoup de boulot.
\end{itemize}


\subsection{Voxel}
Un voxel est un cube 3D (pixel) qui contient des informations (bool in/out, int type\_mat, ...).\\
Le modèle voxel, c'est du Lego (je cite).\\

selon un abus de langage, un voxel, c'est une grille 3D (souvent régulière) : {une box 3D, int n(subdivision régulière de la boite), un tableau de données de taille $n^3$ représentant}.\\

Remarque : tableau 1D pour stocker 3D : index = $i+ nj +n^2k$. $i= index \%n$. j =$((index - i)(n)\%n)$.

Inconvénients : 
\begin{itemize}
	\item cout possiblement très élevé (O$(n^3)$).
	\item données discrètes -> besoin de lisser
\end{itemize}

Avantages :
\begin{itemize}
	\item in/out immédiat
	\item accès rapides aux datas
	\item structure de données très pratiques en simulation
\end{itemize}


Soit $p \in R^3$, comment savoir dans quel cube voxel je suis?\\
$\mat{u \\ v \\ w} = \mat{\frac{p_x - a_x}{b_x - a_x}\\ ....\\ ...} \in [0, 1]^3$\\
$\matrix{i\\j\\k} = \matrix{E(u x n)\\ ...\\...}$.

remarque : optimisation technique
index $= i + nj + n^2k$ peut poser des problèmes d'optimisation. Car accèder à des voxels cote à cote selon i se fait avec un + 1 dans la mémoire, mais accéders à des vocels cote a cote selon j ou k se fait en + n. Et la mémoire n'aime pas. Donc il existe des techniques plus avancées pour gérer ce problème.

-> E.Haine & T.Möller, \textit{Real time Rendering}, 3 eme édition, AK Peters.\\
\textit{Fundamental of CG}, P. Shirley, A.K Petes.


\subsection{CSG}
définir un objet par un arbre de construction:
\begin{itemize}
	\item Noeuds : opérateur booléen (Union, intersection, différence)
	\item Feuilles : primitives géométriques (celles pour lesquelles on peut determiner si on est dedans (volumme))
\end{itemize}

exemple : si on fait l'union d'un cylindre et d'un cube, on obtient une masse, et si on y enlève un cylindre couché, on obtient un marteau.\\

On fait du complexe par assemblage simple.

Requête fondamentale sur un mdoèle volumique : bool inside(pt) -> est-ce qu'un pt est à l'intérieur de mon modèle?

exemple d'implémentation : \\
\begin{verbatim}
N::inside(p) = 0
-> Sphère::inside(p)
-> U::inside(p)
\end{verbatim}
Tout noeud doit savoir si un point est à l'intérieur :
\begin{verbatim}
class N {
	public : bool inside(v) const = 0;
}
class S : public N{
	protected :
		V c;
		d r;
	public :
		bool Inside(v) const;
}
class U : public N {
	protected :
		N *a;	//Data sous arbre
		N *b;
	public :
		bool Inside(v) const;
}

bool S::Inside(p) const {
	return (p-c)(p-c) < r?true:false;
}

bool U :: Inside(p) const {
	return a->inside(P) || b->Inside(p);
}
\end{verbatim}

\begin{itemize}
	\item Cube : six tests
	\item Sphère d(p, c)
	\item Cylindre : d(p, $\delta$)
	\item Tore : d(p, $C$)
	\item Cone : complexe
\end{itemize}

Tester si je suis à l'intérieur d'un maillage est extrêment compliqué. Il faut que le maillage soit fermé, cohérent, et 2-variété (Manifold). On utilise le théorème de Jordan, en tirant une demi-droite partant du point. Si le nbre d'intersection avec le maillage est pair, le point est en dehors, sinon, il est dedans. Maillage::Inside(p) -> Triangle::Inside(p).

\newpage
\section{Texture}
\subsection{Texture and (Shading ou illumination globale)}

Texture : Carac de surface/volumme du matériau, lui donnant sa couleur propre.\\
Illumination globale/Shading : interraction entre la lumière, les matériaux et l'oeil. -> Constitura le chap III

Hypothèse : la couleur est représenté dans un système RougeVertBleu : un certain espace de couleur peut-être approximé par un triplet de valeur $ \in[0, 1]^3$. $[0, 0, 0]$ représente le noir, $[1, 1, 1]$ le blanc.\\

Problème : \\
être capable de définir la couleur $c(p) \forall p \in Obj$. Ce qui nous intéresse plus précisément, c'est la couleur des points à la surface de l'objet : $c(p) \forall p \in Surf(O)$ -> Texture surfacique. Il existe aussi la texture volumique.

Évidement, les texture volumiques sont plus destinées aux modèles volumiques,  mais les textures surfacique peuvent être utilisés avec les deux modèles.\\

Critère de comparaison de méthode de texture : \\
\begin{itemize}
	\item rapidité de calcul
	\item mémoire nécessaire
	\item réalisme -> critère subjective
	\item facilité d'usage/de contrôle
\end{itemize}

\vskip 2cm
\subsection{Placage de texture}
Soit un modèle géométrique G (surface maillée). On cherche une correspondance entre tous les points de la surface (en 3D) et les points de ce que serait cette surface en 2D. On travail dans le domaine de la texture $\omega = [0, 1]^2$, qui contiendra une image de résolution n*n.\\
Pour trouver la correspondance, on travail avec des coordonnées liés à chacun des triangles du modèle géométrique. On rajoute, en prenant par exemple un triangle abc, les coordonnées des points dans la texture -> $a, (u_a, v_a); b(u_b, v_b); c, (u_c, v_c)$.\\

Pour un point p ($ \forall p \in G$, $p \in T_{(abc)} \in \omega$), soit $(\alpha _p, \Beta _p)$ les coordonnées barycentrique de p dans T. $q \in T \in \Theta$ avec les mêmes $(\alpha, \Beta)$.

Problème : \\
Coment découper la peau et la déplier ?

\vskip 1cm
Cas simples : formes géométriques élémentaire :\\
\begin{itemize}
	\item C'est simple pour un plan
	\item Pour une sphère, on utilise les coordonnées polaires des points : $v = \frac{C + \frac{\pi}{2}}{\pi}$ et $v = \frac{\theta}{2\pi}$.\\
\end{itemize}

\vskip 1cm
Cas réels : \\
C'est fait à la main : dépliage des (u, v). Il existe des outils d'aide.

Critique:\\
\begin{itemize}
	\item temps de calcul : temps réel, (GPU), le calcul des (u, v) est fait en amont (préprocessing)
	\item coût mémoire : taille d'une image n*n -> coûteux, car $n \approx 1024$, soit 4 Mo
	\item réalisme : Non applicable
	\item contrôle : placement très précis -> bcp d'opérations à faire à la main pour être très précis
\end{itemize}

\subsection{Textures volumiques}
c(p) va être une fonction math $\R^2 \rightarrow \R^4$\\ $p(x, y, z) \rightarrow (R(x, y, z), V(x, y, z), B(x, y, z), \alpha (x, y, z))$
void Mesh::append(const Mesh & mToAppend) {

Critique : \\
\begin{itemize}
	\item temps de calcul : plus long, mais désormais fait par le GPU Shader -> temps réel désormais
	\item coût en mémoire : une fonction, donc une texture réprésente quelques ko.
	\item réalisme : gros inconvénients : seules quelques textures se prêtent bien à leur éxécution sous la forme d'une fonction
	\item contrôle extrêment difficile : les fonctions ont des paramètres, parfois beaucoup.
\end{itemize}
Elles marchent bien dans des "niches" ou les textures procédurales sont très efficaces -> un certain nombre de matériaux, comme la pierre, un certain nombre de métaux, le crépi, le bois, le béton, qui ont comme point commun d'avoir soit des processus aléatoires, ou au contraire des processus réguliers.

\subsection{Bruit de Perlin}
On a besoin de fonction qui vont "faire" de "l'aléatoire". Travaux de recherches dans les années 82 avec Ken Perlin (Récompensé).\\
On a besoin de f° qui aille de $\R^3$, $\R^2$ ou $\R$ à $[0, 1]$, qui soient lissent et qui soient lisses et aléatoires.\\

Idée : construire une fonction qui passera par des points $(k, f(k))$ ayant une bonne distribution aléatoire pour $f(k)$ et qui soit lisse. On veut une fonction faisant des montagnes russes, et qui soit continue et dérivable.

Values Noise :

Besoin : \\
\begin{itemize}
	\item distribution des valeurs aléatoires $(k, , f(k))$ \\
		\begin{itemize}
			\item fonction mathématiques
			\item tables et permutations
		\end{itemize}
	\item lissage : droite de degré 3.
\end{itemize}
\vskip 1cm
Gradient Noise :\\
On se donne des valeurs aléatoires


\vskip 1cm
Le bruit sert de base à plein de combinaisons :
\begin{itemize}
	\item tâches de rouille sur du métal : \\
		\begin{verbatim}
		Color Rouille( Vector P){
			double n = noise(p);
			//Seuillage
			if(n < 0.8) {
				return Color::Metal;
			}else{
				return Color::Rouille;
			}
		}
		\end{verbatim}
	\item Bois avec irrégularités :\\
		\begin{verbatim}
		Color Bois(Point3f){
			double d = sqrt(p*n*n + p*y*y);
			return mix(Color::Sombre, Color::Clair, cos(d));
		}
		\end{verbatim}
		pb : passer de quelque chose d'extrêment régulier à une forme iréégulière, avec de l'aléatoire. Ce qu'on souhaite faire, c'est de temps en temps, faire une déformation de manière aléatoire : on met un point p dans un point p'.\\
		\begin{verbatim}
			Color BoisIrreg (Point3f){
				Point3f q = p + epsilon * Point3f(noise(p), .., ...);
				return Bois(q);
			}
		\end{verbatim}
		On ne se sert pas directement de noise pour déterminer la couleur, on s'en sert pour faire une transformation
	\item Synthèse multi échelle :\\
		noise(p) prend des échantillons tous les t temps. On souhaite avoir une fonction avec plus de grains, plus irrégulière encore.\\
		$f(p)= a_1 noise(\frac{p}{T_1}) + a_2 noise(\frac{p}{T_2} + a_3 noise(\frac{p}{T_3}) + ...$ où $T_1$ est une période plus grande, pour obtenir une basse fréquence, et $T_2$ une période plus petite, pour obtenir une fréquence moyenne, et ainsi de suite.\\
		$f(p) = \sum _{i = 0} ^{n-1} a_i n (\frac{p}{T_i} + \sigma _i$ où $a_i$ est la i-eme amplitude, $T_i$ est la i-eme période$\sigma _i$ est le i-eme offset.
		C'est le Warping.
\end{itemize}
C'est à la base de beaucoup d'algo de génération de texture volumique.

\subsection{Généralisation}
La texture ne représente pas que la couleur. Il existe bien d'autre paramètres qui serviront à definir la façon dont on doit calculer la couleur en fonction de la lumière :
\begin{itemize}
	\item Le coefficient de réflexion
	\item la brillance
	\item la transparence $\alpha$
	\item la rugosité
	\item ...
\end{itemize}
Pour le placage de texture :\\
En plus de l'image RVB, il faudra des images (map) pour chacun des paramètres => La mémoire nécessaire augmente énormément.

\vskip 1cm
Le bumpMapping :\\
encodage de varation de normales "fines" sur un modèle géométrique "grossier".\\
Prenons un arbre réel, avec de l'écorce avec de très petites variations géométriques, on utilisera énormément de petts triangles ($10^6$). Si on prend un arbre fait de 10 triangles, on estime que l'écorce avec de fines variations est approximativement égale à une écorce lisse, à la différence de la présence de plus de normales. On code alors une fonction qui a perturber la normale.

C'est une utilisation soit d'une fonction, soit d'une carte d'image pour générer de petite variations, en faisant une éconnomie de géométrie.


%Notes sur les textures
% pour avoir une meilleure granularité, faire la somme de bruit à différentes fréquences
% pour avoir des veinures, ajouter une valeur absolue T. On prend la valeur absolue de |f - T| (f1)et la valeur absolue de |T-f|, f la fréquence du signal, qu'on divise par  1 - T (f2). Pour obtenir des pics et le reste aplati, on utilise 1 - (1 - x^2)^2

\newpage
\section{Shading}

\subsection{intro}
Le Shading, c'est simple. Enfin non. Le Shading est l'une des deux techniques qui s'opposent en synthèse pour créer un objet à partir de sa description, de la lumière, etc. \\
L'autre est la global Illumination, qui simule les interractions physiques entre les photons et les matériaux, avec toute l'optimisation des structures de données et des algorithmes pour gérer les M de photons envoyés. Ce ne peut pas être utilisé en temps réel.

Si on veut travailler en temps réel, on ne simule pas, on triche. Le but est que ça fasse réaliste à la fin.

\subsection{Concepts de base}
Deux branches du Shading : réaliste, et NPR/NPS : Non PhotoRealistic Shadering

Modèle de Phong :\\
Pb : voir feuilles

La couleur c du point p de normale $\overrightarrow{n}$ vue par $\overrightarrow{e}$ éclairé par $\overrightarrow{l}$vaut :\\
$ c = C_a + C_d + C_s $ (1)\\
avec $C_a$ la couleur ambiante : $ C_a = cste$ (2),\\
$C_d$ la couleur diffuse : $ C_d = (l\dot n) . m$ (3), très bonne approximation pour les matériaux mats\\
et $C_s$ la couleur spéculaire : $C_s = (r \dot e)^{n_\sigma} . m$, utilisé pour les matériaux brillants. Cet indice contrôle le cône d'éblouissement, choisi en fonction de $n_\sigma$. $r = 2(n\dot l). n - l$

Modèlé de Phong complet (avec couleur) :\\
$c = C_a * C_{al} + n\dot l) C_d * C_{dl} + (e\dot r)^{n_\sigma} C_s * C_{sl}$\\
avec $C_a$ mat, $C_{al}$ light, $C_d$ couleur mat, $C_{dl}$ light, $C_s$ couleur mat, $C_{sl}$ light, \\
avec (par exemple) $C_a * C_{al} = \mat{R_1\\V_1\\B_1}\mat{R_2\\V_2\B_2} = \mat{R_1R_2\\V_1V_2\\B_1B_2\\}$.\\

\vskip 1cm
\subsection{effets spéciaux imposteurs}
Le rendu s'appui nsur le calcul de la couleur des pts d'un triangle avec sa texture et 1 prog de shading. En gros, la seule chose dont on dispose sur une carte graphique, c'est un système capable de coller de petits bout de trinagle colorés un peu partout. Comment gérer tous les effets volumiques? \\
La transparence alpha va permettre une infinité d'effets par application de couches successives. Un imposteur, c'est une géométrie minimale (triangle, quad), avec une apparence en générale très très complexe (Texture, éclairage, transparence complexe).\\
Énorme économie en mémoire.

Type d'effets disponibles pour la géométrie complexe simplifiée :\\
\begin{itemize}
	\item un arbre : Le tronc, ayant un gros volume, est représenté par un maillage, et le feuillage est un entremelement de quad avec des dessins de brances dessus. À l'affichage, les parties transparentes s'effacent, et il ne reste plus que les dessins de branches.
\end{itemize}

On a des programmes qui font des approximations.

\subsections{effets de volume semi transparents}
\begin{itemize}
	\item fumée
	\item feu
	\item glow
\end{itemize}
-> utilisations de nuage d'imposteur. On fait un quad avec une texture au dessin opaque et au reste transparent. Pour une fumée, il s'agit juste de faire monter les quad, en diminuant l'opacité au fur et à mesure que le quad monte.

\subsection{effets de lumière}
On peut gérer les effets de lumière de la même lumière. Par exemple, pour des rayons de lumières traversant une canopée et atteignant des grains de poussière. Normalement, il faudrait calculer la trajectoire des rayons à travers le feuillage. EN utilisant les imposteurs, on utilise des quads représentant les rayons fantomatiques

[...]

\section{Terrains}
\subsection{Data}
Beaucoup de chose possible mais taille grande (O($n^2$)) et traitement idem.
\subsection{Fonctions}

\subsection{Erosion}
SI on regarde bien, un terrain modélisé par des données se prête bien à l'édition. Un terrain modélisé par des fonctions procédurales simule quelque chose de réaliste. Un terrain modélisé par erosion consiste à simuler un phénomène présent dans la nature :
\begin{itemize}
	\item Éolien
	\item Thermique
	\item hydrolique
	\item climatique
\end{itemize}
Tous ces phénomènes sont commentés en géomorpholige. EN informatique, on s'en inspire et on essaye de les imiter. \\

On veut des algos d'imitation des phénomènes -> En Computer Graphics, c'en est une application grossière, et on l'applique à nos structure de données pour obtenir des modèles de terrains un peu plus réaliste.

\textbf{Érosion thermique}\\
Principe : cycle jour nuit avec forte amplitude de température : $\detla \theta$ important sur les reliefs (moins coté nord, plus coté Ouest Est). Une partie du BedRock se transforme en rocher. Sous l'action de la température, ces rochers vont se transformer en sédiment, et dans une deuxième phase, ils seront transportés le long de la pente.\\
Deux phases :
\begin{itemize}
	\item bedrock -> sédiment (arrachement)
	\item sediment -> transport, mvmt. Déplacement tt que l'angle au repos n'est pas atteint. Une fois que le rocher atteint un endroit suffisement plat, il s'arrête. L'angle au repos est l'angle à la base du tas lorsque qu'on accumule des grains.
\end{itemize}

Implémentation : au niveau des structure de données, il ne faut pas travailler avec un modèle monocouche, mais un modèle à deux couches de matériaux.

\begin{verbatim}
class T{
	Vector a, b;
	int n;
	Vector<double> br;	//Vecteur pour la roche
	Vector<double s;	//Vecteur pour le sédiment;
}
\end{verbatim}
SUr tous les pts d'échantillonnage Pi, on a une couche $br_i$, et une surcouche $s_i$.

Au niveau des algos, besoins de deux algo : un qui dise 
\begin{verbatim}
T::ImpactThermique(){
	$\forall_{ij}
	orientation N S E W en ij;
	//plus la pente est élevé, , plus on a une orientation sud, et plus l'impact thermique sera fort. Au contraire, il sera d'autant plus faible avec une plus grande épaisseur de sédiment.
	//Il existe un épuiseur de sed :
	br[i] -= e;
	s[i] += e;
}

T::Stabilize(){
	//Pour trois points d'échantillonnage i, i+1 et i-1, on a trois valeurs de br différentes. Pour savoir si des sédiments doivent tomber, il faut calculer l'angle entre les br adjacents. On peut avoir un déplacement sur les huits voisins.
	$\forall$ i j {
		calculer pour tous les (k, l) des 8 Voisins
			la pente/ l'angle $\alpha$
			tester si éboulement ou non
		si 0 éboulement return;
		choisir \epsilon = cste choisie. On prend \epsilon en ij, et on le répartit sur les voisions proportionnellement à la pente.
		s[i, j] -= \epsilon;
		s[h, l] += fraction(\epsilon);
	}
	//Il faut calculer une map $\delta$ s, puis calculer d'un coup pour cette map $\delta$ s
}
\end{verbatim}
L'érosion thermique modélise relativement bien les accumulutations en rocher en bas de pentes fortes. Cet algo de stabilisation a besoin de plusieurs passes pour stabiliser le terrain. Tant que stabilize signale qu'il faut continuer, on continuer.

\vskip 1cm
\textbf{Érosion hydrolique :}\\
Bcp plus complex, bien meilleurs résultats -> car permet de modéliser les effets du ruissellement, des crêtes et des vallées.\\
Deux stratégies -> simulation temps réel(prise en compte de la dynamique des fluides) OU simulation d'effet géologique (quasi statique -> moyenne par siècle de transmission d'écoulement d'eaux).\\

On rajoute, en plus d'un matériau sédiment, un matériau eau plus ou moins virtuel. Cette eau s'écoule avec une certaine vitesse v. En fonction de la vitesse, il y aura un arrachement en fction de la vitesse, et un dépôt partiel en fonction de la vitesse.\\
Il y a un certain nombre de conditions (présence de bedrock, de sédiment, vitesse de d'écoulement).\\

From Dust(goole).
\vskip 2cm
\subsection{Modèles volumiques}
Pb bcp + coteux en terme de mémoire : data -> 3D -> O($n^3$).\\
Utilisé pour modéliser des grottes, des surplombs, etc. Voxels existent mais ont un cout très très élevé en mémoire.\\
Une solution : pile de matières -> tableau 2D de piles. Utilisation d'une grille, avec, au lieu d'un voxel, on dit qu'il y une certain hauteur de rocher. Puis une certaine hauteur d'air. Puis de nouveau une certaine hauteur de rocher. Cela permet de faire des surplombs, des grottes. Toute l'astuce consiste en une certaine épaisseur d'air.




Conclusion :\\
Perspectives d'avenir -> Terrains 3D (fonctions) ET mutlimatériaux (couche{rochers, branches, feuilles, débris}).\\
Deuxième gros défi : terrains immenses avec précisions. Comment générer des kilomètres carrés de terrains à la précsion du centimètre.


\vskip 2cm
\subsection{Végatation}
$\approx$ arbre.

Intro :\\
arbes et herbes (buissons et arbustes assimilés aux arbres)\\
\begin{itemize}
	\item Modélisation des formes des plantes (forme et texture) -> modélisation  + animation de la végétation(vent)\\
	\item modélisation de la répartition/distribution. -> simulation d'écosystème\\
\end{itemize}

EN fait, ces deux phénomènes sont liés.\\

Modélisation :\\
deux gros problèmes :
\begin{itemize}
	\item Quelle structure de données?
	\item comment créer/générer
\end{itemize}

\paragraph(Structure de Données)
Maillages plus ou moins précis.

Selon le niveau de détail, on peut avoir :
\begin{itemize}
	\item 1 Quad avec une image d'arbre dessus, qu'on fait tourner face à la caméra
	\item 2 Quad composant un arbre, sans avoir besoin de la faire tourner, simulant un arbre un tant soit peu plus réaliste. $=< 4 $ Triangles max.\\Imposteurs
	\item Un base Mesh low Poly pour le tronc (200 T), et 100 Quad pour le feuillage -> Billboard cloud. Il est possible de monter à 1000 ou 2000T.
	\item Un high Poly pour le tronc (très précis, quelques 100k Triangles), et on modélise 1 Quad par feuille (de l'ordre de 40 000 feuilles), ou mieux, une feuille maillée. Facilement $10^6$ Triangles.
\end{itemize}
Les premiers modèles sont généralement statiques. Le dernier est adapté à l'animation.

\paragraph{Génération}
\begin{itemize}
	\item Édition à la main : + long, + précis, capable de conserver un certain style
	\item SpeedTree : référence dans la création d'arbre pour le temps réel
	\item Création automatique : comment créer de la masse de façon générique.
\end{itemize}

Comment faire de la création automatique d'arbre :
\begin{itemize}
	\item L-Systems (Lindemeyer 1965) : un arbre respecte un processus de croissance; on l'imite cette structure par 1 algorithme/grammR.\\
		Axiome : tige |\\
		Règle : tiges | -> Y\\ 
		et ainsi de suite. On peut enrichir la gramR de paramètre -> gramR paramétrée.\\
		On peut y intégrer des choix aléatoires -> gramR stochastique : une gramR peut offrir une variété. Deux arbres ne seront jamais exactement les mêmes.\\
		gramR ouverte : la gramR n'est pas fermée, elle dépend d'une requête vers l'extérieur, sur l'environnement. On bloque la génération si il n'est pas possible de générer des branches quelques part.\\
		+ varié, + générique, + automatique, + adapté à l'émulation, - gramR, + bio conforme/inspiré (CIRAD/CRIRAD Montpellier).\\
		pb : Comment passer de la structure en battonnet à une surface de tronc réaliste? Non résolu.
	\item 
\end{itemize}

\paragraph{Comment distribuer de la végétation}
\begin{itemize}
	\item à la main
	\item automatiquement :
		\begin{itemize}
			\item procédural -> règles ad hoc de placement.\\
				En général, sur un terrain, on définit une zone (cercle) dans laquelle on place la forêt. Pour la générer automatiquement au sein de ce cercle :
				\begin{itemize}
					\item solution 1 : \begin{verbatim}
						for(i = 0; i < n; i ++)	// générer n arbres
						{
							a[i] = Cercle(c, r).randomInside();
						}
					\end{verbatim}
					Pb : arbres trop proches, arbres sur des zones trop pentues.
				\item solution 2 : faire un tirage en ne créant a[i] que s'il n'est pas trop proche de a[k] avec k < i.\\
					Faire un check de pente sur le terrain.
				\item moduler le développement de l'arbre a[i] à son environnement.
				\end{itemize}
				Il s'agit de regarder l'environnement avant de créer un arbre.
			\item simulation écologique : évolution d'une population
		\end{itemize}
\end{itemize}
Pb : comment faire cohabiter des espèces?\\
Exemple : une zone partagée entre des sapins(denses), et des platanes (denses).\\
Si par exemple, les platanes sont arrivés en premier, alors les sapins écrasent les platanes.\\

Simulation d'écosystème :\\
Distribution sous la forme d'un jeu de la vie : règles de developpement, de reproduction et de mort des plantes.\\
En général, 1 itération = 1 an.


Pour chaque type/espèce d'arbre, et à chaque itération :\\
Chaque arbre a un cercle avec un rayon (ressources nécessaire). L'arbre peut :
\begin{itemize}
	\item mourir, en fonction de l'espérance de vie, avec des lois de probabilités indiquant ses chances de survie.
	\item se reproduire, et où. En général, dans un rayon Rs autour de lui, de façon aléatoire.
	\item tester sa compétion avec ses voisins, sa compatibilité avec son environnement. Cela influe sur ses capacités.
\end{itemize}

\section{Structures de données accélatrices}
Un certain nmbre de modèle de maillages créés : On souhaite pouvoir accélérer les traitements à tout moments.

Objectif : Optimisation
\begin{itemize}
	\item visualisation -> n'afficher que ce qui est nécessaire (visible ds le cone de la caméra)
	\item collision -> ne tester que ce qui est nécessaire (proche)\\
		On s'appuie sur des techniques de structuration de l'espace : traitement des objets un par un. Gains souvent titanesques (facteur 10 puissance 2, 3)
\end{itemize}

\subsection{Classification}
Hiérachie 
\begin{itemize}
	\item Hiérachies Volumes englobants
		\begin{itemize}
			\item AABB : des boîtes à chaussures qui contiennent des boîtes à chaussures (Axis Aligned B.. Boxes)
			\item OBB : des boites d'orientation quelconques -> On évite, car rarement utiles
			\item BVH : des cercles qui contiennent des carrés qui contiennent des triangles .. -> Pas adpaté
		\end{itemize}
	\item Division de l'espace
		\begin{itemize}
			\item BSP : arbre binaire -> La classification nécessite des calculs (trois additions, trois multiplication, deux additions, et un test) ce qui peut couter cher en grand nbre.
			\item k-D-tree : nbre fini de directions\\
				k-D-tree particulier : utilisation des plans principaux du repère uniquement -> ce sont une optimisation des BSP. Le cout de classification est un test. SI on peut l'utiliser, on l'utilise en le préférant aux BSP
		\end{itemize}
		\item les grilles
			\begin{itemize}
				\item les grilles régulières -> division simples -> très utilisées car elles ont de bonnes propriétés de vitesses. Elles bouffent bcp de mémoire 
				\item les grilles adaptatives (Quad-Tree, Oct-Tree) qui sont la version récursives d'une division par quatre en 2D et par 8 en 3D
				\item les multigrilles -> structure récursive de division sans forcément de régularité -> une multigrille bien choisie permet d'éviter les pertes de performances d'un oct-tree
			\end{itemize}
\end{itemize}

\subsection{BVH AABB}
But : structurer l'espace en un certain nbre de région incluses les unes dans les autres de manières hiérarchiques
En informatique, cela correspond à un arbre sous la forme
\begin{itemize}
	\item  B12345
		\begin{itemize}
			\item B12
				\begin{itemize}
					\item B1 -> O1
					\item B2 -> O2
				\end{itemize}
			\item B345
				\begin{itemize}
					\item B3 -> O3
					\item B4 -> O4
					\item B5 -> O5
				\end{itemize}
		\end{itemize}
\end{itemize}

\vskip 2cm

\textbf{Algorithme :}\\
Test p $\in$ O
\begin{verbatim}
bool N ::Test(p)
	if(terminal) return O.appartien(p);
	else
		si(p n'appartient pas à B) return false
		sinon return left::Test(p) || right::Test(p)
\end{verbatim}
Autre exemple : $\delta \inter O$\\
L'algo est similaire, en remplaçant juste le test sur l'appartenance du point à la boite par tester si la droite passe par la boite englobante.

\vskip 2cm
\textbf{Construction}
Pb : quelle hiérachie optimise pour n objets.\\
C'est un problème NP complet => utilisaton d'heuristiques de construction (ou construction à la main).\\
Par exemple, on choisit une direction de construction (X, Y) -> On partage la population en 2 parties de tailles plus ou moins égales. On on récursive, mais en changeant de direction (en tournant orthognonalement).\\
Très efficace -> calculs sur VE : boites AABB sont rapides, et constructions rapides

Remarques : certains objets rentrent mal dans des AABB

plutot que de faire un volume Englobant + complex, faire l'union de volumes englobants simples.

Remarque : Les hiérachies de boites englobantes sont utiles pour des objets statiques ou à mouvement régulier et limité dans l'espace


\subsection{Grilles régulières}
(dessin d'un grille)

une grille est un tableau à N dimmensions (2 ou 3 en fonction du nbre de dimmensions dans lesquels on travaille). Les cases contiennent la liste des objets touchant leur case.

\vskip 2cm
\textbf{Algo} p $\in$ O\\
trouver case(i, j, k)\\
tester les objets de (i, j, k)\\

Algoithme de Bresonamm : très efficace, et est capable de donner l'ordre de passage à travers les cases.

\vskip 2cm
\textbf{construction}
Les cas ou une grille n'apporte quasiment rien :
\begin{itemize}
	\item les cas ou ts les objets ou presque sont dans une même case. -> la grille n'a rien pu séparer : distribution / dispersion hétérogène
	\item distribution hétérogènes de petits objets, et gros objets sur plusieurs cases -> tailles différentes
\end{itemize}

Grilles= bien si objets $\approx$ de meme taille et bien distribués

A utiliser pour la foret.

Remarque : 2 stratégies :
\begin{itemize}
	\item chaque case stocke $\forall$ les objets qui les touchent. 1 objet est référencé quatre fois au maximum si son diamètre est inférieur à la taille d'une cellule.
	\item Comme le nbre de cas expliqué ci dessus peut être grand, on peut associer un objet à une et une seul case. On asscoie la case qui contient le centre/point d'ancrage de l'objet à cet objet : loose grid (grille relachée). On teste également le 1 voisinage pour l'appartenance d'un pt.
\end{itemize}


\subsection{Structures particulières}
Ttes ces structures sont très générales.\\

$\exists$ structure spécifiques à des pbs particuliers.

Exemple : les portals\\
C'est une extension des structures précédentes pour les scènes architecturales. Permet d'accéléer au max la visualisation dans les scènes d'intérieur.\\
Avec les structures précédentes, on devait afficher ttes les pièces qui intersectent le champ de vision. Or elles peuvent être dissimulés par des murs.\\
Les portals divisent en plusieurs sections numérotés, et rajoutent des "portals" entre des sections (les ouvertures, fenetres, portes). On récurse l'affichage uniquement si la vue intersecte un portal. On peut limiter l'affichage d'une section vue par un portal en récursant par un volume V de visualisation plus petit, dépendant de la taille de l'ouverture. On récurse V sur les ouvertures de la section.

Il peut y avoir bcp de structures et d'algos spécifiques. Plus il y a d'hypothèses fortes, plus on peut s'appuyer dessus pour faires des structures spécifiques adaptées.
\end{document
